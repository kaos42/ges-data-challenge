
Some forms of mass transit, such as rail-based systems, exist within a
closed environment, and traffic accidents are prevented by human
dispatchers. Other forms, however, depend upon infrastructure used by
heterogeneous populations. In particular, buses operate on municipal
roadways, often in dense urban areas. Bus-automobile interactions pose
unique problems to public safety, especially for public and school
buses, which stop frequently. Other problems for bus safety include
bus-pedestrian collisions, and an outsized impact of dangerous weather
conditions. All told, in 2014, 22000 persons were injured, and 281
died, in bus accidents \citep{truckfacts}.

As a mass transit solution, bus systems experience unique
challenges. To begin with, the bus system operates within the larger
system of public-access roads. This poses problems to the entire
community, starting with pedestrians, who account for nearly 25\% of
bus accident fatalities, as well as drivers of other cars (26\%),
small truck drivers (18\%), and bus passengers (16\%). Furthermore, the
problem is concentrated on buses operating in urban or suburban areas:
73\% of bus fatalities occur in school buses or transit buses, as
opposed to long-distance bus lines \citep{truckfacts}. In order to
promote the welfare of all individuals impacted by bus accidents, it
is essential to understand the causes and characteristics of traffic
accidents involving buses. Such an understanding will provide
explanations for the causes of bus accidents, which has policy
relevance for traffic control, transit system design, and
distracted-driver regulations.

A taxonomy of bus accidents would provide insight into the variety of
conditions that frequently occur simultaneously in each accident
subtype.  These conditions could be very specific to each subtype,
allowing policymakers to focus their attention on relevant regulations
that need to be changed. We believe that such taxonomies should be
data-driven to avoid subjectivity, and discover novel subtypes of
accident.

The construction of data-driven bus accident taxonomies is not
widespread in the traffic safety literature. The leading study of
\cite{prato2013bus} employs data collected nearly a decade ago. By
applying their method to more recent data, we can understand how the
taxonomy itself has changed from the 2005--2009 period to the
2010--2015 period.

The National Highway Traffic Safety Administration (NHTSA) annually
publishes detailed datasets of auto accidents, which include variables
describing the events that caused the accident, demographic
information about any drivers or pedestrians involved, and other
categorical variables describing the incident. The latent structure of
accidents in the NHTSA data can be used to identify subpopulations of accidents to
guide a taxonomic study of risk factors of bus accidents.

However, the large volume of observations and high-dimensionality of
variables makes it difficult to characterize bus accidents by manual
exploratory data analysis. Therefore, in this paper we constructed a
taxonomy of bus accidents using cluster analysis. Our analysis
discriminates several subpopulations of bus accidents on the basis of patterns of shared
attributes, which provide compelling qualitative profiles of these subpopulations.

We aggregated a subset of variables in the NHTSA's General Estimates
System (GES), and considered crashes from two separate time periods,
2005--2009, and 2010--2015. We find clusters representing
subpopulations of bus accidents that are broadly consistent across the
two datasets, but we also note the differences in the clusters across
the datasets, which gives us insight into how the taxonomy itself
changed over the two time periods.

The paper is organized as follows. In section \ref{sec:data}, 
we describe the NHTSA's national crash data sample, an ongoing data-collection
initiative, which is the dataset used in our analysis.
In section \ref{sec:methods}, we describe the two-stage clustering approach
used to build the taxonomy. In section \ref{sec:results}, we 
take a close look at the clusters and interpret them to understand the 
distinct subpopulations of bus accidents formed by the taxonomy. Section \ref{sec:discussion}
discusses possible implications of our analysis and addresses some of its limitations.
